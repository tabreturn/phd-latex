\chapter{Abstract}

%\todoinline{\textit{Thesis must include an extended abstract of no more than 500 words; this abstract is shared with potential examiners, so must offer a clear summary of all aspects of the thesis.}}

Traditional approaches to teaching programming fundamentals often pose barriers for learners whose primary orientation is artistic- or design-focused. This thesis investigates Python-based creative computing environments as a solution, evaluating existing tools and developing new software and techniques for visual learning contexts, with a focus on students in creative and interdisciplinary fields.

The thesis adopts a hybrid folio/thesis-by-publication format, encompassing software development, scholarly publications, presentations, learning resources, and creative works. Collectively, these outputs demonstrate scholarly contribution, active community engagement, and the artistic applications of novel programming practices.

A central contribution is the creation of Thonny-py5mode, a plugin that integrates the py5 library into the Thonny IDE, serving as a contemporary successor to Processing's discontinued Python Mode (Processing.py). It enables learners to generate graphical and interactive output by writing Python code, bridging technical concepts with creative exploration. An empirical evaluation applies the Task--Technology Fit framework to assess Thonny-py5mode, surveying 143 students in an introductory Python course and analysing their responses with Structural Equation Modelling (SEM) to capture factors influencing user adoption in educational contexts.

The thesis also considers the emerging role of generative AI in programming education. While acknowledging its benefits, it highlights risks of over-reliance and proposes graphical task design with Thonny-py5mode as a strategy to mitigate GenAI misuse. As a relatively late addition, this strand of inquiry foregrounds future research directions and underscores the continuing evolution of the field.

This research demonstrates how Python tools and purpose-built environments, supported by targeted curricula, can lower barriers to programming education by employing graphical and multimedia output to enhance student engagement, conceptual understanding, and motivation.
