\chapter{Creative Works}
\label{chap:creative-works}

This chapter presents creative works generated using Python-based creative coding tools and techniques developed through this PhD research. Each output has been exhibited at an event and/or featured in a publication and, in all instances, selected through a review process that validates significance and quality. Alongside the outputs in Chapters \ref{chap:publications} (\nameref{chap:publications}) and \ref{chap:presentation-outputs} (\nameref{chap:presentation-outputs}), these works constitute the PhD folio and support three of the five research objectives: \hyperref[ro:tool-development]{Tool Development}, \hyperref[ro:creative-outputs]{Creative Outputs}, and \hyperref[ro:broader-dissemination]{Broader Dissemination}.

Traditional artistic practice often engages with personal, cultural, or philosophical themes. In contrast, the works in this chapter focus on pushing the boundaries of tools and exploring new forms of computationally-driven visual expression. As Brown and Sorensen suggest, creative artefacts in practice-led research serve as both a method of inquiry and an outcome, enabling a feedback loop between experimentation and reflection~\cite{smith_practice-led_2009}. More accurately, this research's iterative, exploratory process treated creative artefacts as byproducts of experimentation rather than standalone artistic pieces. This reflects a tradition of artists creating custom tools to expand creative boundaries, challenging the limitations of existing instruments to explore new possibilities in digital media~\cite{national_endowment_for_the_arts_tech_2021}. As Paul writes---

\begin{quote}
Code has also been referred to as the medium, the `paint and canvas,' of the digital artist, but it transcends this metaphor in that it even allows artists to write their own tools---to stay with the metaphor, the medium in this case also enables the artist to create the paintbrush and palette.~\cite{paul_codedoc_2002}

\end{quote} 

Accordingly, the chapter contents evidence the creative potential of Thonny-py5mode and other Python tools realised through creative work, rather than focusing on traditional artist statements typical of gallery or student projects.

Inspired by open-source principles and hacker cultures, the conceptualisation phases, tool development, and coding practices adopted agile processes, where software and techniques evolved through experimentation, testing, and community feedback~\cite{yoon_networked_2018}. Similar to Processing or p5.js, environments like Thonny-py5mode embody the dual role of enabling innovation while serving as artefacts of creative exploration.

The chapter organises content into distinct sections, with each dedicated to a specific output (constituting one or many creative works), presented in chronological order and concluding with the most recent. The individual creative works presented here remain largely untitled; instead, names such as ``\nameref{sec:north}'' serve as project labels, where experimentation encapsulated threads exploring particular themes and techniques. For clarity and consistency, each section provides a short account of the following aspects:

\begin{itemize}

% the INTRO & WHY
\item \textbf{Description \& Context}: 
Outlines the work's format, thematic focus, and dissemination platform, identifying any technical, creative, or conceptual challenges that it addressed. 

% the HOW
\item \textbf{Creative Process}: 
Briefly describes tools, techniques, and methods used in making the artwork(s), with emphasis on the Python-based creative coding approaches central to this research. 

% the SO WHAT
\item \textbf{Impact \& Reflection}: 
Summarises the work's influence on subsequent outputs, reflecting on artistic, technical, and conceptual learnings, and where these insights shaped the research.

\end{itemize}

Additionally, each write-up includes \textbf{visual documentation}, which may comprise final outputs, process work, and photographs illustrating key stages of development.

While the creative works in this chapter pursue varied technical and aesthetic directions, they all serve the same research purpose: to iteratively explore, test, extend, inform, and refine the tools, methods, and pedagogical strategies developed through this PhD. The primary function of this chapter is to document the material evidence and process work of this endeavour. \textbf{Accordingly, it provides a deliberately concise treatment of the descriptions, details of creative process, and reflective commentary for each output}, leaving the \nameref{chap:publications} and \nameref{chap:presentation-outputs} chapters to contextualise and elaborate on the broader insights and implications. 

%This includes (1) demonstrating the innovative use of Python-based creative coding environments, tools, and techniques; (2) highlighting the artistic and technical achievements enabled by these tools; (3) inspiring further exploration and experimentation within the creative coding community; and (4) contributing practical examples to academic discourse on advancing Python programming education through creative computing environments.

%%%%%%%%%%%%%%%%%%%%%%%%%%%%%%%%%%%%%%%%%%%%%%%%%%%%%%%%%%%%%%%%%%%%%%%%%%%%%%%
%%%%%%%%%%%%%%%%%%%%%%%%%%%%%%%%%%%%%%%%%%%%%%%%%%%%%%%%%%%%%%%%%%%%%%%%%%%%%%%

\cleardoublepage
\section{North}
\label{sec:north}

\vspace{1.5em}
\begin{outputmeta}
\textbf{Exhibition/Event Name}: ADA Network Symposium 2021 \\
\textbf{Author(s)}: Bunn, T. \\
\textbf{Venue}: Wellington, New Zealand: Massey University College of Creative Arts \\
\textbf{Start/Finish Dates}: 04 December 2021\\
\textbf{URL}: \url{https://ada.net.nz/tag/symposium2021} \\
\textbf{Acceptance Letter/Invitation or Event Programme}: See Appendix \ref{appendix:ada-2021}
\end{outputmeta}

\subsection{Description \& Context}

These works consisted of generative pen plotter drawings created with Python code and rendered in coloured inks on a range of paper stocks. They featured as part of ADA's 10th symposium, \textit{Indeterminate Infrastructures}, which examined how creative practices engage with the uncertain architectures of contemporary media systems. ADA (Aotearoa Digital Arts Network) connects New Zealand artists, curators, educators, and the public to advance digital and electronic art through collaboration, exhibitions, and education. 

\subsection{Creative Process}

The workflow relied almost entirely on Python, utilising Thonny-py5mode for coding and executing sketches, with Inkscape solely for sending plotting instructions. The code generated wave- and ripple-based mathematical patterns, producing spherical and radial line formations that shifted between structured order and chaotic variation. Transferring precise vector coordinates into ink through a pen tip and onto a textured paper surface introduced subtle irregularities and `failures,' adding further layers of visual complexity and interest to the plotted output.

\subsection{Impact \& Reflection}

This series contributed to the PhD research by demonstrating how Thonny-py5mode and SVG Python libraries can support pen-plotting workflows for generative art. Beyond generating line art, the project tested new approaches to layering SVG markup (for colour switching) and vector path optimisation. The resulting insights informed several of the outputs discussed in the \nameref{chap:presentation-outputs} chapter---most directly the talks detailed in Sections \ref{sec:generate-svg-for-pen-plotters-using-python} and \ref{sec:generative-art-with-python-using-py5-and-bpy}.

\begin{figure}[ht]
\centering
\includegraphics[width=1.0\textwidth]{chapters/chapter06-creative-works/north-process}
\caption{Plotting spherical line formations generated using Thonny-py5mode. Photos by the author.}
\label{fig:north-process}
\end{figure}

\begin{figure}[H]
\centering
\includegraphics[width=1.0\textwidth]{chapters/chapter06-creative-works/north}
\caption{A finalised plot from the \textit{North} series. Metallic inks on black card. Artwork by the author.}
\label{fig:north-three}
\end{figure}

%%%%%%%%%%%%%%%%%%%%%%%%%%%%%%%%%%%%%%%%%%%%%%%%%%%%%%%%%%%%%%%%%%%%%%%%%%%%%%%
%%%%%%%%%%%%%%%%%%%%%%%%%%%%%%%%%%%%%%%%%%%%%%%%%%%%%%%%%%%%%%%%%%%%%%%%%%%%%%%

\cleardoublepage
\section{Destruction of Kepler-186f}
\label{sec:the-end-of-random-seed}

\vspace{1.5em}
\begin{outputmeta}
\textbf{Exhibition/Event Name}: \{Between\} 2021 \\
\textbf{Creator(s)}: Bunn, T. \\
\textbf{Venue}: Coimbra, Portugal: Nest Collective \\
\textbf{Start/Finish Dates}: 08 December 2021 -- 10 December 2021 \\
\textbf{URL}: \url{https://pcdcoimbra.dei.uc.pt/2021/exhibition/between} \\
\textbf{Acceptance Letter/Invitation or Event Programme}: See Appendix \ref{appendix:between}
\end{outputmeta}

\subsection{Description \& Context}

A generative, plotted artwork exhibited at \textit{\{Between\} An Inventory of Anachronic Practices}, an international exhibition themed around anachronism. Panellists received 60 submissions in response to an open call and, through a double-blind review process, selected 24 artworks to exhibit. This piece speculated on humanity's precarious place in the universe, presenting intelligent life as likely elsewhere, and positioning our technological and societal state as a potential \textit{anachronism} in cosmic time. Specifically, this final poster depicted the annihilation of Kepler-186f, one of multiple generative outputs that visualised the destruction of some planet.

\subsection{Creative Process}

Again, this work employed a generative coding approach, utilising Python code within the Thonny-py5mode environment. Its algorithm relied on hash-based seeding functions---in this case, a deterministic integer derived from the MD5 hash of the string \texttt{Kepler-186f}. Each unique string/planet name generated a distinct arrangement of ten panels depicting the annihilation of a that planet, producing outcomes that appeared chaotic yet remained reproducible.

\subsection{Impact \& Reflection}

The project further experimented with techniques for generating SVG markup, informing several of the outputs discussed in the \nameref{chap:presentation-outputs} chapter---most directly the talks detailed in Sections \ref{sec:generate-svg-for-pen-plotters-using-python} and \ref{sec:generative-art-with-python-using-py5-and-bpy}. 

Creatively, it investigated programmatic approaches to comic panel layout and engaged with procedurally generated pictorial narratives (or ``silent'' comics).

\begin{figure}[htbp]
\centering
\includegraphics[width=0.9\textwidth]{chapters/chapter06-creative-works/the-end-of-random_seed-process}
\captionsetup{width=0.9\textwidth}
\caption{Generative plot depicting the destruction of Kepler-186f. Black fine-tip pen on white paper. Photo by the author.}
\label{fig:the-end-of-random_seed-process}
\end{figure}

\begin{figure}[htbp]
\centering
\bigskip
\includegraphics[width=0.5\textwidth]{chapters/chapter06-creative-works/the-end-of-random_seed}
\captionsetup{width=0.5\textwidth}
\caption{\textit{Destruction of Kepler-186f}. Exhibited at \textit{\{Between\}}, 2021. Artwork by the author.}
\label{fig:the-end-of-random_seed}
\end{figure}

%%%%%%%%%%%%%%%%%%%%%%%%%%%%%%%%%%%%%%%%%%%%%%%%%%%%%%%%%%%%%%%%%%%%%%%%%%%%%%%
%%%%%%%%%%%%%%%%%%%%%%%%%%%%%%%%%%%%%%%%%%%%%%%%%%%%%%%%%%%%%%%%%%%%%%%%%%%%%%%

\cleardoublepage
\section{South}
\label{sec:south}

\vspace{1.5em}
\begin{outputmeta}
\textbf{Exhibition/Event Name}: LINK 2021 \\
\textbf{Author(s)}: Bunn, T. \\
\textbf{Venue}: Auckland, New Zealand: Auckland University of Technology \\
\textbf{Start/Finish Dates}: 01 December 2021 -- 03 December 2021 \\
\textbf{URL}: \url{http://dx.doi.org/10.24135/link2021.v2i1.167} \\
\textbf{Acceptance Letter/Invitation or Event Programme}: See Appendix \ref{appendix:link-2021}
\end{outputmeta}

\subsection{Description \& Context}

A series of generative artworks exploring wave motion through pen-plotter drawing, presented at the LINK Conference. Established in 2019, LINK is a platform for dialogue and knowledge exchange, particularly across the Global South, aiming to create links between diverse practices, contexts, and epistemologies in art and design. South extends techniques developed in the \nameref{sec:north} series. 

\subsection{Creative Process}

Whereas the \nameref{sec:north} works sought to establish a Thonny-py5mode workflow for plotter art, this series foregrounded experimentation. Figure~\ref{fig:south-three} tested the limits of the 2-axis plotter machine, burning out servo motors through highly detailed and prolonged plotting. The series also incorporated experiments with moiré patterns (Figure~\ref{fig:south-process}, top right) and with layering plotter drawings over prints of generative raster imagery (Figure~\ref{fig:south-process}, bottom right).

\subsection{Impact \& Reflection}

The series reflected continued experimentation with Python-based creative coding techniques, feeding back into Thonny-py5mode development and informing several of the outputs discussed in the \nameref{chap:presentation-outputs} chapter---most notably the talks detailed in Sections \ref{sec:generate-svg-for-pen-plotters-using-python} and \ref{sec:generative-art-with-python-using-py5-and-bpy}, as well as aspects discussed in Section~\ref{sec:thonny-+-py5-a-python-3-environment-for-processing}. 

As with earlier plotter works, it demonstrated the translation of algorithmic techniques into material outcomes, foregrounding the variability inherent in such processes. This includes the deliberate use of pens that intermittently dried out before resuming, visible in the red lines at the top right of Figure~\ref{fig:south-process}.

\begin{figure}[htbp]
\centering
\includegraphics[width=1.0\textwidth]{chapters/chapter06-creative-works/south-three}
\caption{Black fine-tip pen on white paper. Approximate plotting time: 12~hours. Artwork by the author.}
\label{fig:south-three}
\end{figure}

\begin{figure}[htbp]
\centering
\includegraphics[width=1.0\textwidth]{chapters/chapter06-creative-works/south-process}
\caption{Plotting moiré patterns (top right), and over raster prints (bottom right). Photos by the author.}
\label{fig:south-process}
\end{figure}

%%%%%%%%%%%%%%%%%%%%%%%%%%%%%%%%%%%%%%%%%%%%%%%%%%%%%%%%%%%%%%%%%%%%%%%%%%%%%%%
%%%%%%%%%%%%%%%%%%%%%%%%%%%%%%%%%%%%%%%%%%%%%%%%%%%%%%%%%%%%%%%%%%%%%%%%%%%%%%%

\cleardoublepage
\section{Relics U+130C8}
\label{sec:relics-u130c8}

\vspace{1.5em}
\begin{outputmeta}
\textbf{Exhibition/Event Name}: Motyf 2022 \\
\textbf{Author(s)}: Bunn, T. \\
\textbf{Venue}: Warsaw, Poland: PJAIT \\
\textbf{Start/Finish Dates}: 25 March 2022 -- 15 April 2022 \\
\textbf{URL}: \url{https://motyf2021.webflow.io/#motyf_exhibition} \\
\textbf{Acceptance Letter/Invitation or Event Programme}: See Appendix \ref{appendix:motyf-21-22}
\end{outputmeta}

\subsection{Description \& Context}

A generative plotter-drawing series exhibited at Motyf: an international symposium and media art exhibition held biennially. Motyf 2021 (delayed to 2022 due to the pandemic) centred around the theme ``Communicating Complexity.'' This work marked a shift away from mechanical to more organic forms, seeking to emulate hand-drawn qualities and processes of gradual unravelling---suggesting that complexity is best understood through its vulnerabilities as much as its structures.

\subsection{Creative Process}

Again, this series employed Python code written and run through the Thonny-py5mode environment, intentionally employing randomness and decay to introduce irregularities, resulting in robotically plotted drawings that straddle the line between algorithmic precision and human-like mark-making. This included experimenting with the broadest markers the plotter could handle (Figure \ref{fig:relics-artworks-process}, left) and the simulation of roughly masked areas (Figure \ref{fig:relics-artworks}).

\subsection{Impact \& Reflection}

The series reflected ongoing experimentation with diverse Python-based creative coding techniques, feeding back into \textit{Thonny-py5mode} development and informing several of the talks discussed in the \nameref{chap:presentation-outputs} chapter---most notably those in Sections~\ref{sec:generate-svg-for-pen-plotters-using-python} and~\ref{sec:generative-art-with-python-using-py5-and-bpy}, as well as aspects of Section~\ref{sec:thonny-+-py5-a-python-3-environment-for-processing}.

Beyond technical exploration, \textit{Relics U+130C8} examined how generative systems might mimic, diverge from, or potentially hybridise with human drawing practices.

\begin{figure}[htbp]
\centering
\includegraphics[width=1.0\textwidth]{chapters/chapter06-creative-works/relics-artworks}
\caption{Coloured pens on grey card. Accuracy decreases progressively as the rows of circles advance rightward and downward. Artwork by the author.}
\label{fig:relics-artworks}
\end{figure}

\begin{figure}[htbp]
\centering
\includegraphics[width=1.0\textwidth]{chapters/chapter06-creative-works/relics-artworks-process}
\caption{Experimenting with plotting more `human-like' markings. In effect, machines emulating humans emulating a machines. Photo by the author.}
\label{fig:relics-artworks-process}
\end{figure}

%%%%%%%%%%%%%%%%%%%%%%%%%%%%%%%%%%%%%%%%%%%%%%%%%%%%%%%%%%%%%%%%%%%%%%%%%%%%%%%
%%%%%%%%%%%%%%%%%%%%%%%%%%%%%%%%%%%%%%%%%%%%%%%%%%%%%%%%%%%%%%%%%%%%%%%%%%%%%%%

\cleardoublepage
\section{Digital Aquatics}
\label{sec:digital-aquatics}

\vspace{1.5em}
\begin{outputmeta}
\textbf{Publication}: Processing Community Catalog 2001--2021 \\
\textbf{Author(s)}: Processing Foundation; Editors: Reas, C., and McCarthy, L. \\
\textbf{Details}: Hardcover, ISBN-13: 978-0999881323  \\
\textbf{Publication Date}: January 2022 \\
\textbf{URL}: \url{https://archive.org/details/processing-community-catalog-2021} \\
\textbf{Acceptance Letter/Invitation or Event Programme}: See Appendix \ref{appendix:processing-community-catalog}
\end{outputmeta}

\subsection{Description \& Context}

A contribution to the \textit{Processing Community Catalogue}, published to commemorate two decades of Processing and its related initiatives, including its JavaScript, Python, Ruby, and Raspberry Pi variants. This work is a Thonny-py5mode adaptation of a sketch that transforms mathematical formulas into playful, amoeba-like forms. Additionally, it featured in promotional materials for the catalogue's international open call: \url{https://twitter.com/ProcessingOrg/status/1439283507679797250}.

\subsection{Creative Process}

The Python code adapted Johan Gielis' Superformula, building on Lieven Menschaert's NodeBox project\footnote{~\url{https://nodebox.net/code/index.php/Aquatics}} \textit{Aquatics!}. Each run of the sketch generated a unique organism with varying shapes, colours, and other visual details. No two creatures are alike. However, they always possess at least three eyes, may (or may not) sprout hair along their outlines, and can appear to sway in currents simulated by directional noise.

\subsection{Impact \& Reflection}

Again, this work employed Python code written and run through Thonny-py5mode, demonstrating how this approach can produce outputs that evoke a humorous and whimsical feel. 

It also represents a bridge between Processing.py techniques (Sections~\ref{sec:no-starch}, \ref{sec:processing.py-creative-coding-with-python}, \ref{sec:processing-python-mode-for-creative-coding-and-teaching}) and py5, demonstrating that Thonny-py5mode can reliably replace Processing.py. In doing so, it establishes Thonny-py5mode's credibility as a Python~3 successor to Processing's discontinued Python Mode.

\begin{figure}[htbp]
\centering
\includegraphics[width=0.8\textwidth]{chapters/chapter06-creative-works/digital-aquatics}
\captionsetup{width=0.8\textwidth}
\caption{Greyscale artwork published in the \textit{Processing Community Catalog 2001–2021} (printed on lilac paper), inspired by Lieven Menschaert's NodeBox script \textit{Aquatics!}. Artwork by the author.}
\label{fig:digital-aquatics}
\end{figure}

%%%%%%%%%%%%%%%%%%%%%%%%%%%%%%%%%%%%%%%%%%%%%%%%%%%%%%%%%%%%%%%%%%%%%%%%%%%%%%%
%%%%%%%%%%%%%%%%%%%%%%%%%%%%%%%%%%%%%%%%%%%%%%%%%%%%%%%%%%%%%%%%%%%%%%%%%%%%%%%

\cleardoublepage
\section{Chapter Summary}

This chapter has presented a portfolio of creative works developed over the course of the PhD, collectively demonstrating how Thonny-py5mode facilitated novel approaches to generative drawing, while the processes and outputs, in turn, informed its ongoing development. Together, these works contribute practical examples to academic discourse on advancing Python programming education through creative computing environments, and can inspire further exploration and experimentation within the creative coding community.

These works substantively advance three of the thesis' \nameref{sec:research-objectives}: showcasing the development and application of custom Python-based tools; evidencing the production of credible creative outputs; and extending dissemination and engagement through international exhibitions and publications.

This chapter deliberately maintained concise descriptions, deferring to the \nameref{chap:publications} and \nameref{chap:presentation-outputs} chapters to contextualise and elaborate on the broader insights and implications of the creative work. These subsequent discussions demonstrate the innovative use of Python-based creative coding environments, tools, and techniques, and highlight the artistic and technical achievements these solutions enable.
