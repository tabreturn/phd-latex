\subsection{Abstract}

Coding environments that integrate graphical output into programming instruction can enhance engagement, comprehension, and motivation among beginners. This study evaluates Thonny-py5mode, a plugin for the Thonny IDE that emulates the Java-based Processing creative coding IDE, but instead supports Python via the py5 library. Guided by the Task--Technology Fit (TTF) framework, we extend the model with personalised learning, hedonic motivation, and effort expectancy to capture factors influencing Thonny-py5mode user adoption in educational contexts. The study analyses survey data from 143 students in a 100-level university Python course that incorporated four weeks of Thonny-py5mode use, including an assessment requiring participants to recreate six graphical challenges. We employ Structural Equation Modelling (SEM) to test nine hypotheses mapped to our extended TTF framework. Our results show that Thonny-py5mode's usability and task support, along with its ease of use, contributed to a strong perceived TTF, which in turn influenced students' intention to continue using it. These results highlight Thonny-py5mode design priorities that can encourage adoption and sustained use in related educational coding environments and tools. 

\bigskip

\noindent\textbf{Keywords}: Task--technology fit (TTF), Python, Introductory programming, Integrated development environment 

\subsection{Introduction}

Programming graphical output through textual code can make learning to program more accessible and engaging by providing immediate visual feedback. Environments supporting visual output encourage experimentation, deepen conceptual understanding, and boost learner motivation~\cite{bakar_development_2019, balter_enjoying_2010, chiodini_two_2025}. One of the earliest examples is Logo's Turtle graphics, developed in the 1970s at MIT, where learners controlled an on-screen turtle to draw shapes~\cite{solomon_history_2020}. Inspired by this approach, Python has included a Turtle module since version 1.5.2, released in 1999~\cite{hunt_python_2019, python_software_foundation_python_1999}. Building on this tradition, environments such as Processing and p5.js introduce programming through graphical output but rely on Java or JavaScript~\cite{funk_coding_2024, mccarthy_getting_2015, shiffman_nature_2024}. With Processing's Python mode (Processing.py) now unmaintained and incompatible with current IDE versions~\cite{feinburg_python_2023}, a gap exists for educators seeking Processing-based learning experiences in Python. 

Python remains a leading choice for introductory programming due to its readability and industry relevance~\cite{ezenwoye_what_2018, guo_python_2014}, yet many courses emphasise text-only output, focusing on syntax and logic through the console rather than graphical or interactive elements~\cite{greenberg_processing_2016, guzdial_introduction_2005}. This emphasis can limit engagement for learners from creative or interdisciplinary backgrounds; by contrast, Processing-style environments have been shown to enhance understanding, retention, and motivation among these audiences~\cite{levin_code_2021, reas_processing_2006, scherer_vpython_2000} suggesting potential benefits for Python-based tools with similar capabilities. 

This study forms part of the ongoing research and development of Thonny-py5mode (\url{https://github.com/tabreturn/thonny-py5mode}), a bespoke plugin that introduces the Processing creative-coding paradigm to Python through the beginner-friendly Thonny IDE via the py5 library~\cite{annamaa_introducing_2015, schmitz_welcome_2021}. The plugin addresses the gap left by Processing.py, enabling students to produce graphical and interactive output in Python without the complexity of managing manual package installations. Instead, the setup process is handled entirely through Thonny's built-in plugin manager in just a few clicks. Thonny-py5mode's interface (Figure \ref{fig:thonny-py5mode-processing-comparison}) is deliberately designed to emulate the Processing IDE, with a similar layout, colour scheme, built-in sketch runner, and syntax highlighting. A dedicated ``py5'' menu provides quick access to the py5 reference, a colour mixer, an online/printable cheatsheet, and the sketch folder. 

\begin{figure}[htbp]
\centering
\includegraphics[width=1.0\textwidth]{chapters/chapter04-publications/jise/thonny-py5mode-processing-comparison}
%\captionsetup{width=0.8\textwidth}
\caption{Top: Thonny-py5mode plugin activated, running a script. Bottom: Processing 4.x running a script. Screenshots by the author.}
\label{fig:thonny-py5mode-processing-comparison}
\end{figure}

Thonny-py5mode provides switchable coding modes, including an \textit{Imported} mode, which allows learners to use familiar Processing functions such as \texttt{draw()} and \texttt{square()} without explicitly adding Python \texttt{import} statements or \texttt{py5} prefixes. When running a script (or ``sketch''), Thonny-py5mode automatically manages the output window while preserving full Shell compatibility for debugging. Guided by the TTF framework~\cite{dwivedi_task-technology_2012}, we surveyed students in an introductory Python course to evaluate Thonny-py5mode's usability, task support, and overall effectiveness. 

\subsection{Background}
\label{subsec:background}

Goodhue and Thompson (1995) introduced the Task--Technology Fit (TTF) model to explain how well a technology supports users in performing their tasks, and how this alignment influences both system use and performance outcomes. They argued that users achieve better results and are more likely to continue using a system when its capabilities align with the demands of the task~\cite{marikyan_task-technology_2023, goodhue_task-technology_1995}. In this view, \textit{fit} occurs when a tool's features directly support the requirements of the work, enabling more effective task completion. TTF extended earlier adoption models including the Technology Acceptance Model (TAM) and Theory of Planned Behaviour (TPB) by incorporating the direct relationship between task--technology alignment and task performance, which those models did not explicitly address~\cite{ajzen_theory_2020, davis_user_1989}. By taking a post-adoption perspective, TTF highlights that positive attitudes alone do not guarantee improved performance or sustained use, unless the technology fits the task. Goodhue and Thompson (1995) formalised this idea through the ``Technology-to-Performance Chain,'' which posits that performance gains occur only when a strong TTF and actual system use occur~\cite{goodhue_understanding_1995}. 

Education encompasses diverse activities such as instruction, collaboration, and assessment. It employs technologies ranging from LMSs to coding platforms, providing multiple scenarios for TTF analysis. However, simply adopting educational technology is insufficient; tools must align with pedagogical needs to generate meaningful outcomes~\cite{yadegaridehkordi_task-technology_2016}. Several recent studies have applied the TTF framework to evaluate how well digital tools support teaching and learning. For example, Alyoussef (2021) found that higher TTF in MOOCs improved satisfaction, performance, and sustained use~\cite{alyoussef_massive_2021}. Yaakop \textit{et al.} (2020) linked strong alignment with collaborative and resource-sharing tasks to continued engagement with online environments~\cite{yaakop_examining_2020}. TTF studies of Google Workspace have shown that matching tool features to collaborative learning tasks predicts adoption~\cite{budiartha_using_2024, siek_investigating_2022}. Wang and Kartika Sari (2024) demonstrated that aligning game mechanics with learning goals enhanced engagement and reduced cognitive load~\cite{wang_examining_2024}. Lim and Lee (2021) connected fit to creative and collaborative skills development~\cite{lim_extended_2021}. Together, these studies confirm TTF as a valuable framework for evaluating educational technologies, where alignment between tool features and learning tasks may drive engagement, performance, and sustained use. 

Although previous research has proven that the fit between learning task requirements and technological characteristics is a central antecedent of technology adoption, these studies overlooked students' personal characteristics. This omission weakened the effectiveness of the TTF and reduced its explanatory power, since technology must not only meet task requirements but also align with individual learners' traits~\cite{bere_applying_2018}. Researchers have observed similar limitations in TPB, which emphasises attitude, subjective norm, and perceived behavioural control, but has faced criticism for neglecting individual-level characteristics beyond its core constructs~\cite{ajzen_theory_2020, tzeng_predicting_2022}. Beyond generic adoption models, studies in South African higher and basic education underscore that curriculum-level technology integration and educators' ICT self-efficacy meaningfully shape how they appropriate tools in practice \cite{rambe_role_2016, mlambo_effects_2020}.


For these reasons, the current study extends TTF by introducing a personal characteristics construct, operationalised through personalised learning, hedonic motivation, and effort expectancy. Incorporating these traits strengthens the framework and aligns it with the enriched TPB approach, which has shown to outperform the original TPB in predicting behavioural intentions~\cite{ajzen_theory_2020, tzeng_predicting_2022}.

\subsection{Hypothesis Development}

Figure \ref{fig:ttf-diagram-jise} presents our extended TTF model, integrating nine hypotheses. To better frame these, we broadened the \textit{Personal Characteristics} construct to include \textit{Personalised Learning}, \textit{Hedonic Motivation}, and \textit{Effort Expectancy}---factors shown to influence learners' technology engagement and capture individual differences in perceived fit~\cite{shahzad_what_2025}. These highlight pedagogical affordances of Thonny-py5mode that go beyond functional alignment, incorporating insights from acceptance models such as TAM and UTAUT/2, where tailored experiences, perceived enjoyment, and ease of use are recognised as critical drivers of behavioural intention~\cite{bere_empirical_2016, mosunmola_adoption_2018}. 

\begin{figure}[htbp]
\centering{}
\smallskip
\includegraphics[width=1.0\textwidth]{chapters/chapter04-publications/jise/ttf-diagram-jise}
\caption{Extended TTF model for Thonny-py5mode. Constructed by the authors.}
\label{fig:ttf-diagram-jise}
\end{figure}

The subsections below describe each construct and its associated hypotheses, while Section \ref{jise:method} details the analysis method. 

\subsubsection{Task Characteristics}

% hypothesis encircle command
\newcommand{\Hyp}[1]{%
  \tikz[baseline=(X.base)]{
    \node[draw,circle,minimum size=1.8em,inner sep=0pt] (X)
      {\sffamily\bfseries\footnotesize H#1};}}
\newlist{hyplist}{itemize}{1}
\setlist[hyplist]{labelwidth=2em, labelsep=0.5em, leftmargin=!, itemsep=0.5em}
\newcommand{\hyitem}[1]{\item[\Hyp{#1}]}

\textbf{Physical and cognitive features of a task such as complexity, structure, frequency, interdependence, and cognitive load that shape performance and determine how well technology can support it.} In the TTF model, better alignment between task requirements and technology capabilities leads to improved performance, satisfaction, and adoption~\cite{goodhue_task-technology_1995, hidayat_contemporary_2021}. In this study, Python programming tasks include writing and debugging code, incorporating py5 drawing functions, and solving graphical challenges, which require rapid feedback, low cognitive load, and ease of use---capabilities Thonny and Thonny-py5mode aim to provide. The following hypotheses have been formulated based on the task characteristic construct relationship with hedonic motivation and task--technology fit, respectively.

\begin{hyplist}
\hyitem{1} Task Characteristics positively influence the Hedonic Motivation associated with Thonny-py5mode-supported learning. 
\hyitem{4} Task Characteristics positively influences the Task--Technology Fit associated with Thonny-py5mode-supported learning. 
\end{hyplist}

\subsubsection{Technology Characteristics}

\textbf{Functional features, properties, and usability aspects of a system that determine how effectively it supports users' tasks,} including reliability, accessibility, data quality, interactivity, and alignment with user abilities and needs~\cite{delone_delone_2003, goodhue_task-technology_1995}. In the TTF framework, strong alignment between technology characteristics and task requirements enhances performance, adoption, and satisfaction; while poor alignment can hinder task completion, particularly in education~\cite{park_exploring_2019}. In this study, Thonny-py5mode's beginner-friendly interface, graphical output, and real-time debugging aim to meet the needs of Python novices, lowering cognitive load and supporting tasks such as py5 coding, visualisation, and problem-solving. The following hypotheses have been formulated based on the technology characteristic construct relationship with effort expectancy and task--technology fit, respectively. 

\begin{hyplist}
\hyitem{2} Technology Characteristics positively influences the Effort Expectancy associated with Thonny-py5mode-supported learning.
\hyitem{5} Technology Characteristics positively influences the Task--Technology Fit associated with Thonny-py5mode supported learning. 
\end{hyplist}

\subsubsection{Personal Characteristics}

\textbf{Individual user traits, motivations, preferences, and cognitive capabilities that can affect how a user perceives and interacts with technology, as well as how they perform tasks.} These attributes shape perceptions of fit, ease of use, and enjoyment, and can ultimately affect technology adoption and outcome~\cite{dwivedi_task-technology_2012, tamilmani_extended_2021, venkatesh_user_2003, wang_examining_2024}. As Figure \ref{fig:ttf-diagram-jise} illustrates, we operationalise personal characteristics through three constructs: \textit{Personalised Learning}, \textit{Hedonic Motivation}, and \textit{Effort Expectancy}, examined in detail in the three subsections that follow. 

\subsubsection{Personalised Learning}

\textbf{Tailoring the environment to an individual's needs, preferences, and pace of learning to enhance engagement, satisfaction, and performance.} In TTF terms, personalisation enables system features and tasks to adapt to learners' cognitive abilities, motivations, and learning styles~\cite{chao_factors_2019, melzer_conceptual_2019}. Thonny-py5mode supports this through optional tool panels, real-time visual feedback, and coding activities that can be approached at varying levels of complexity, allowing learners to progress at their own speed and focus on relevant concepts. By aligning with students' cognitive and emotional needs, these features can increase engagement and enjoyment in programming, particularly for novice learners~\cite {alyoussef_e-learning_2021, skoric_exploring_2021}. In resource-constrained contexts, higher ICT self-efficacy supports more constructive classroom uses of technology, suggesting that personalisation, which builds confidence, can strengthen fit \cite{mlambo_effects_2020}. The following hypotheses have been formulated based on the relationship between the personalised learning construct and task characteristics, as well as task-technology fit.

\begin{hyplist}
\hyitem{3} Personalised Learning positively influences the Task Characteristics associated with Thonny-py5mode-supported learning. 
\hyitem{6} Personalised Learning positively influences the Task--Technology Fit associated with Thonny-py5mode supported learning. 
\end{hyplist}

\subsubsection{Hedonic Motivation}

\textbf{The extent to which a technology is perceived as enjoyable or fun to use.} Higher enjoyment can increase users' intention to adopt the technology and lead to their improved task performance~\cite{ayasrah_exploring_2020, cheng_role_2020}. Within the TTF framework, hedonic motivation can also promote engagement and reduce cognitive resistance in learning contexts~\cite{thabet_exploring_2024}. In this study, Thonny-py5mode's visual feedback, interactive coding environment, and immediacy aim to create an enjoyable learning experience that sustains interest---particularly for novice programmers---enhancing satisfaction and encouraging continued use~\cite{chao_factors_2019, wang_examining_2024}. The following hypothesis has been formulated based on the relationship between the hedonic motivation construct and task--technology fit. 

\begin{hyplist}
\hyitem{7} Hedonic Motivation positively influences the Task--Technology Fit associated with Thonny-py5mode-supported learning. 
\end{hyplist}

\subsubsection{Effort Expectancy}

\textbf{The perceived ease of use of a technology; alongside TTF, it is a key construct for user acceptance and can influence performance in educational contexts.} TTF suggests that the easier a technology is to use, the better its fit to the task, the stronger the learning outcomes~\cite{alyoussef_massive_2021, ayasrah_exploring_2020, skoric_exploring_2021}, and the lower the resistance to adoption~\cite{abbad_using_2021}. In this study, Thonny-py5mode is designed to offer high perceived ease of use through a simplified, learner-friendly interface, clear error messages, and near-instant visual feedback. These features minimise the effort required for novices to complete coding tasks effectively and confidently~\cite{annamaa_introducing_2015, annamaa_thonny_2024, chao_factors_2019, skoric_exploring_2021}. The following hypothesis has been formulated based on the effort expectancy construct relationship with task--technology fit.

\begin{hyplist}
\hyitem{8} Effort Expectancy positively influences the Task--Technology Fit associated with Thonny-py5mode supported learning. 
\end{hyplist}

\subsubsection{Task--technology Fit}

\textbf{Describes the degree of alignment between a technology's capabilities and the requirements of a user's task.} High fit is typically associated with positive outcomes such as productivity, satisfaction, and behavioural intention to use the technology~\cite{goodhue_task-technology_1995, wang_examining_2024}. As Figure \ref{fig:ttf-diagram-jise} illustrates, it encompasses three dimensions in our model: \textit{Task Characteristics}, \textit{Technology Characteristics}, and \textit{Personal Characteristics}. In this study, Thonny-py5mode-supported learning demonstrates task--technology fit when these three dimensions align, meeting the demands of graphical Python programming tasks and the needs of novice learners. The following hypothesis has been formulated based on the task--technology fit construct relationship with behavioural intention.

\begin{hyplist}
\hyitem{9} Task--technology fit positively influences students' Behavioural Intention to learn using Thonny-py5mode. 
\end{hyplist}

\subsubsection{Behavioural Intention}

\textbf{A person's willingness or intent to use a technology in the future, over a sustained period.} This is a core psychological variable in technology acceptance models such as UTAUT and UTAUT2, where it not only reflects interest but also predicts actual system use. Several aspects shape behavioural intention, including performance expectancy, effort expectancy, and TTF~\cite{chao_factors_2019, abbad_using_2021, wang_examining_2024}. Higher behavioural intention is associated with greater engagement, improved performance, and stronger adoption continuity~\cite{thabet_exploring_2024, cheng_role_2020}. In Thonny-py5mode-supported learning, beginner-friendly features, graphical feedback, extendable libraries, and a low learning curve (with the capacity to support more advanced projects over time) aim to address both task and personal needs, which should increase students' willingness to adopt and continue using it~\cite{ wang_examining_2024}.

\subsection{Course Setting}

This study evaluated the perceptions of Torrens University Australia (TUA) students using Thonny-py5mode in \textit{ITP122 Introduction to Programming}: a 12-week Bachelor level course introducing Python to beginners, delivered in Trimester 1, 2025. Students used Thonny-py5mode during Weeks 5--8, after completing Assessment 1 (in Week 4) and leading into Assessment 2 (due at the end of Week 8). In the first four weeks, they worked in IDEs such as Jupyter Notebook, PyCharm, and Spyder, producing only console output. Assessment 2 required them to complete graphical coding challenges using Thonny-py5mode. Figure \ref{fig:ttf-assessment-2-tasks} presents the six assessment tasks, to be recreated as accurately as possible using Python code. 

\begin{figure}[htbp]
\centering
\includegraphics[width=0.8\textwidth]{chapters/chapter04-publications/jise/ttf-assessment-2-tasks}
\captionsetup{width=0.8\textwidth}
\caption{Thonny-py5mode Assessment 2 challenges. Developed by the author.}
\label{fig:ttf-assessment-2-tasks}
\end{figure}

These tasks required students to apply \texttt{if-else} statements, and \texttt{for} and \texttt{while} loops, in combination with a limited set of approximately 15 py5 drawing functions, with additional marks awarded for clear structure and effective commenting. The weekly content leading up to Assessment~2 covered: (Week 1) Introduction to Computer Programming and IDEs; (Week 2) Fundamentals of Decision Logic; (Week 3) Intermediate Decision Logic; (Week 4) Lists and Dictionaries; (Week 5) Nested Decision Logic; (Week 6) Simple Loops; (Week 7) Basic Functions; and (Week 8) Intermediate Loops. 

\subsection{Method}
\label{jise:method}

Our proposed model (Figure \ref{fig:ttf-diagram-jise}) examined key factors shaping novice learners' behavioural intention to learn Python programming using Thonny-py5mode. Our study method employed a quantitative, survey-based approach, selected for its confirmatory and deductive qualities, and common application in studies applying TTF and other technology acceptance frameworks~\cite{alyoussef_e-learning_2021, budiartha_using_2024, wang_examining_2024}. The model involved testing a priori hypotheses and examining theoretically grounded relationships using data from authentic educational settings~\cite{dwivedi_task-technology_2012, park_exploring_2019, mosunmola_adoption_2018}. 

We applied a structured validation approach to develop and validate the measurement instrument: a survey questionnaire comprising seven sections (Table \ref{tbl:measurement-items-for-the-study}) aligned with our model constructs (see Figure \ref{fig:ttf-diagram-jise}). We first developed our model through an extensive literature review to identify and define theoretical constructs, refining these through specialist input and analysis. Experts in Python teaching and learning reviewed the draft questionnaire to improve its structure and wording. Based on feedback, we amended potentially ambiguous statements to establish the instrument's content validity~\cite{budiartha_using_2024, yaakop_examining_2020}. 

We recruited ITP122 participants aged 18 years or older from TUA's campuses in Adelaide, Brisbane, Melbourne, Sydney, and online. From Week 9, following the submission of Assessment 2, we invited students to complete a voluntary, anonymous questionnaire either online or on paper, with no incentives offered and assurances that participation would not affect their grades. All items used a 5-point Likert scale (1 = strongly disagree to 5 = strongly agree) addressing constructs across the study's extended TTF model. Table \ref{tbl:measurement-items-for-the-study} outlines the model constructs alongside their corresponding item/indicator groupings.

\begin{table}[!htbp]
  \centering
  \fontsize{9.75pt}{10pt}\selectfont{}
  \renewcommand{\arraystretch}{1.8}
  \begin{tabular}{
    >{\raggedright\arraybackslash}p{\dimexpr 0.15\linewidth-2\tabcolsep} % Construct
    >{\raggedright\arraybackslash}p{\dimexpr 0.85\linewidth-2\tabcolsep} % Items for Modelling
  }
  \hline
  \textbf{Construct} & \textbf{Items for Modelling} \\
  \hline
  \smallskip{} \textit{Task Characteristics (TC)} & 
  \begin{itemize}[leftmargin=*, topsep=0pt, partopsep=0pt]
    \item Thonny-py5mode supports effective hands-on learning through Python graphics creation.
    \item Programming graphics using Thonny-py5mode encourages critical and analytical skills.
    \item Overall, Thonny-py5mode is an effective environment for learning Python fundamentals.
  \end{itemize} \\
  \hline
  \smallskip{} \textit{Technology Characteristics (TCS)} &
  \begin{itemize}[leftmargin=*, topsep=0pt, partopsep=0pt]
    \item Thonny-py5mode features an intuitive, user-friendly interface.
    \item I can complete graphical coding tasks efficiently using Thonny-py5mode.
    \item Thonny-py5mode provides reliable, stable performance during coding sessions.
  \end{itemize} \\
  \hline
  \smallskip{} \textit{Personalised Learning (PL)} &
  \begin{itemize}[leftmargin=*, topsep=0pt, partopsep=0pt]
    \item Using Thonny-py5mode to code graphical output supports learning at my own pace.
    \item I can customise the Thonny-py5mode environment to match my learning preferences.
    \item Thonny-py5mode accommodates different learning styles and strategies.
    \item Overall, Thonny-py5mode promotes a personalised learning experience.
  \end{itemize} \\
  \hline
  \smallskip{} \textit{Hedonic Motivation (HM)} &
  \begin{itemize}[leftmargin=*, topsep=0pt, partopsep=0pt]
    \item Thonny-py5mode allows me to express my creativity through coding.
    \item Learning to code graphical output with Thonny-py5mode is fun and enjoyable.
    \item Thonny-py5mode motivates me to experiment with different coding techniques and ideas.
    \item Overall, coding graphics with Thonny-py5mode is a positive and enjoyable experience.
  \end{itemize} \\
  \hline
  \smallskip{} \textit{Effort Expectancy (EE)} &
  \begin{itemize}[leftmargin=*, topsep=0pt, partopsep=0pt]
    \item Thonny-py5mode is easy to install and set up.
    \item Thonny-py5mode is easy to learn for beginners.
    \item Thonny-py5mode makes it easy to debug code.
    \item Overall, Thonny-py5mode is efficient and easy to use.
  \end{itemize} \\
  \hline
  \smallskip{} \textit{Task--Technology Fit (TTF)} &
  \begin{itemize}[leftmargin=*, topsep=0pt, partopsep=0pt]
    \item Thonny-py5mode makes it easy for me to learn new coding concepts.
    \item Thonny-py5mode provides useful and efficient debugging tools.
    \item Overall, Thonny-py5mode supported the enhancement of my coding skills.
  \end{itemize} \\
  \hline
  \smallskip{} \textit{Behavioural Intention (BI)} &
  \begin{itemize}[leftmargin=*, topsep=0pt, partopsep=0pt]
    \item I intend to continue using Thonny-py5mode for future projects.
    \item I would recommend Thonny-py5mode to others learning to code.
    \item I expect to use Thonny-py5mode frequently for regular graphics and interactive coding.
  \end{itemize} \\
  \hline
  \end{tabular}
  \caption{Measurement items for the study}
  \label{tbl:measurement-items-for-the-study}
\end{table}

The researchers conducting the study did not teach or grade the ITP122 course, and students were aware that the Thonny-py5mode was under evaluation. The online questionnaire closed one week after the 12-week course concluded, and we distributed final paper questionnaires for completion during Week 12 classes. Completion time averaged 10--15 minutes, observed in both in-class and online administration. All participants provided informed consent, and the study complied with research ethics guidelines approved by the TUA Human Research Ethics Committee (HREC; Application 0394), classified as Low and Negligible Risk (LNR). 

We employed Structural Equation Modelling (SEM), which requires an adequate number of participants to ensure consistent and reliable results. Guidelines for SEM recommend that the minimum sample size should exceed the number of distinct parameters in the model, with a common rule of thumb of 5--10 observations per indicator~\cite{hair_multivariate_2019, hair_introduction_2021}. The conceptual model in this study contained 24 indicators, suggesting an appropriate sample size between 120 and 240 participants. We collected data from 157 respondents. Data screening procedures addressed missing values, outliers, normality, and multicollinearity~\cite{hair_multivariate_2019}, resulting in the removal of 14 cases. Table \ref{tbl:study-demographics} presents an overview of the screened respondent statistics, comprising 143 valid responses, exceeding the minimum (120) recommended for SEM and thereby supporting the reliability of the analyses. 

\begin{table}[!htbp]
  \centering
  \fontsize{9.75pt}{10pt}\selectfont{}
  \renewcommand{\arraystretch}{1.8}
  \begin{tabular}{
    *{4}{>{\raggedright\arraybackslash}p{\dimexpr 0.25\linewidth-2\tabcolsep}}
  }
  \hline
  \textbf{Measure} & \textbf{Item} & \textbf{Percentage} & \textbf{Frequent} \\
  \hline
  Participant Campus 
  & Melbourne & 23.78\% & 34 \\
  & Sydney & 20.28\% & 29 \\
  & Adelaide & 10.49\% & 15 \\
  & Brisbane & 09.09\% & 13 \\
  & Online & 36.36\% & 52 \\
  \hline
  ITP122 Gender 
  & Male & 63.64\% & \\
  & Female & 36.36\% & \\
  \hline
  ITP122 Age
  & 18--22 & 49.65\% & \\
  & 23--27 & 32.87\% & \\
  & 28--32 & 12.59\% & \\
  & 33+ & 04.90\% & \\
  \hline
  \end{tabular}
  \caption{Study demographics}
  \label{tbl:study-demographics}
\end{table}

We further examined the dataset for common method bias using Harman's single-factor test and the marker variable test~\cite{hair_introduction_2021}. Harman's test showed that no single factor accounted for the majority of variance, and the marker variable results indicated that substantive construct variance exceeded any method-related variance. Together, these findings suggest a low likelihood of common method bias.

\subsection{Data Analysis and Results}

Following data preparation and bias checks, we applied SEM to test and validate the research model, examining the critical determinants of students' intention to learn Python programming using the Thonny-py5mode coding environment. To empirically test the validity and reliability of the constructs presented in Figure~\ref{fig:ttf-diagram-jise}, we conducted a Confirmatory Factor Analysis (CFA) in AMOS version~26 using the survey data. The goodness-of-fit (GOF) statistics assessed were the likelihood ratio chi-square ($\chi^{2}$), the ratio of $\chi^{2}$ to degrees of freedom ($\chi^{2}/df$), the root mean square error of approximation (RMSEA), and the Comparative Fit Index (CFI). Table~\ref{tbl:measurement-model-validation-and-reliability-statistics} reports the measurement model evaluation results.

\begin{table}[!htbp]
  \centering
  \fontsize{9.75pt}{10pt}\selectfont{}
  \renewcommand{\arraystretch}{1.8}
  \begin{tabular}{
    >{\raggedright\arraybackslash}p{\dimexpr 0.175\linewidth-2\tabcolsep} % Construct
    >{\raggedright\arraybackslash}p{\dimexpr 0.095\linewidth-2\tabcolsep} % Item
    *{8}{>{\centering\arraybackslash}p{\dimexpr 0.09125\linewidth-2\tabcolsep}} % FL--alpha
  }
    \hline
    \textbf{Construct} & \textbf{Item} & \textbf{FL} & \textbf{IR} & \textbf{CR} & $\boldsymbol{\chi^2/\mathrm{df}}$ & \textbf{p} & \textbf{CFI} & {\scriptsize \textbf{RMSEA}} & $\boldsymbol{\alpha}$ \\
    \hline
    \multirow[t]{3}{=}{Task Characteristics (TC)} 
    & TC1 & 0.84*** & 0.71 & 0.93 & 1.67 & 0.14 & 0.98 & 0.06 & 0.91 \\
    & TC2 & 0.89*** & 0.79 &       &      &      &      &      &      \\
    & TC3 & 0.91*** & 0.83 &       &      &      &      &      &      \\
    \hline
    \multirow[t]{3}{=}{Technology Characteristics (TCS)} 
    & TCS1 & 0.83*** & 0.70 & 0.90 &      &      &      &      & 0.90 \\
    & TCS2 & 0.79*** & 0.68 &      &      &      &      &      &      \\
    & TCS3 & 0.85*** & 0.73 &      &      &      &      &      &      \\
    \hline
    \multirow[t]{4}{=}{Personalised Learning (PL)} 
    & PL1 & 0.77*** & 0.65 & 0.88 &      &      &      &      & 0.86 \\
    & PL2 & 0.83*** & 0.70 &      &      &      &      &      &      \\
    & PL3 & 0.86*** & 0.74 &      &      &      &      &      &      \\
    & PL4 & 0.86*** & 0.74 &      &      &      &      &      &      \\
    \hline
    \multirow[t]{4}{=}{Hedonic Motivation (HM)} 
    & HM1 & 0.77*** & 0.75 & 0.89 &      &      &      &      & 0.85 \\
    & HM2 & 0.73*** & 0.63 &      &      &      &      &      &      \\
    & HM3 & 0.84*** & 0.71 &      &      &      &      &      &      \\
    & HM4 & 0.81*** & 0.68 &      &      &      &      &      &      \\
    \hline
    \multirow[t]{4}{=}{Effort Expectancy (EE)} 
    & EE1 & 0.94*** & 0.88 & 0.94 &      &      &      &      & 0.92 \\
    & EE2 & 0.92*** & 0.82 &      &      &      &      &      &      \\
    & EE3 & 0.88*** & 0.77 &      &      &      &      &      &      \\
    & EE4 & 0.91*** & 0.81 &      &      &      &      &      &      \\
    \hline
    \multirow[t]{3}{=}{Task Technology Fit (TTF)} 
    & TTF1 & 0.79*** & 0.68 & 0.98 &      &      &      &      & 0.86 \\
    & TTF2 & 0.86*** & 0.74 &      &      &      &      &      &      \\
    & TTF3 & 0.83*** & 0.70 &      &      &      &      &      &      \\
    \hline
    \multirow[t]{3}{=}{Behavioural Intention (BI)} 
    & BI1 & 0.84*** & 0.71 & 0.93 &      &      &      &      & 0.91 \\
    & BI2 & 0.94*** & 0.88 &      &      &      &      &      &      \\
    & BI3 & 0.92*** & 0.85 &      &      &      &      &      &      \\
    \hline
    \multicolumn{2}{l}{\textbf{Recommended value}} & $\geq$ 0.70 & $\geq$ 0.50 & $\geq$ 0.50 & $\leq$ 0.30 & $\geq$ 0.05 & $\geq$ 0.90 & $\leq$ 0.08 & $\geq$ 0.70 \\
    \multicolumn{10}{l}{\footnotesize Note(s): ***$p \leq 0.001$, **$p \leq 0.01$, *$p \leq$ 0.05} \\
    \hline
  \end{tabular}
  \caption{Measurement model validation and reliability statistics}
  \label{tbl:measurement-model-validation-and-reliability-statistics}
\end{table}

We analysed convergent validity using the factor loading (FL) value and the composite reliability (CR) value. The general rule is that the FL and CR values should be at least 0.50, and preferably 0.70 or higher, with all FLs statistically significant. Following this rule, we dropped five items in the original conceptual model~\cite{hair_multivariate_2019}. Other items with FLs ranged from 0.70 to 0.94, and constructs with CRs above 0.88, as shown in Table \ref{tbl:measurement-model-validation-and-reliability-statistics}, indicating a high convergent validity. 

Construct reliability testing consists of item reliability (IR) and construct reliability~\cite{hair_multivariate_2019}. IR is determined by the squared multiple correlation value, with values above 0.50 indicating acceptable reliability. The IR values for all the items are higher than 0.50, indicating that these are sufficient for measuring the construct. We evaluated the construct reliability by calculating Cronbach's alpha ($\alpha$) coefficient, with a value of 0.70 or higher considered acceptable~\cite{hair_multivariate_2019}. Six constructs have high Cronbach's alpha coefficients, all above 0.85, indicating strong construct reliability.

We assessed the GOF statistics for the final measurement model after conducting validity and reliability tests. The normalised chi-square value ($\chi^2/df$ = 1.67) fell below the recommended cut-off of 3.00, indicating that the model did not significantly differ from the observed data. The RMSEA value (0.06) was below the 0.08 threshold, and the CFI value (0.98) exceeded the 0.90 threshold. Together, these results indicate a good fit between the model and the data, making it suitable for hypothesis testing.

Discriminant validity is supported when the square root of the AVE for each construct is greater than its correlations with other constructs~\cite{hair_multivariate_2019}. Table \ref{tbl:ave-and-squared-correlation-matrix} presents the correlation matrix for the seven constructs. All constructs demonstrated high discriminant validity, with AVE values ranging from 0.77 to 0.85. For example, the AVE for Effort Expectancy (0.85) exceeds its correlations with other constructs, which range from 0.22 to 0.77, indicating strong discriminant validity for this construct. 


\begin{table}[!htbp]
  \centering
  \fontsize{9.75pt}{10pt}\selectfont
  \renewcommand{\arraystretch}{1.8}
  \begin{tabular}{
    >{\raggedright\arraybackslash}p{\dimexpr 0.125\linewidth-2\tabcolsep} % empty corner
    *{7}{>{\raggedright\arraybackslash}p{\dimexpr 0.125\linewidth-2\tabcolsep}}
  }
  \hline
  & \textbf{TC} & \textbf{TCs} & \textbf{PL} & \textbf{HM} & \textbf{EE} & \textbf{TTF} & \textbf{BI} \\
  \hline
  TC   & \textbf{0.82} \\
  TCs  & 0.56 & \textbf{0.79} \\
  PL   & 0.48 & 0.42 & \textbf{0.84} \\
  HM   & 0.62 & 0.52 & 0.47 & \textbf{0.82} \\
  EE   & 0.39 & 0.49 & 0.53 & 0.38 & \textbf{0.85} \\
  TTF  & 0.51 & 0.44 & 0.41 & 0.36 & 0.28 & \textbf{0.77} \\
  BI   & 0.45 & 0.38 & 0.49 & 0.42 & 0.41 & 0.22 & \textbf{0.81} \\
  \hline
  \end{tabular}
  \caption{AVE and squared correlation matrix}
  \label{tbl:ave-and-squared-correlation-matrix}
\end{table}

We assessed the overall fit of the structural model using multiple fit indices to evaluate the proposed hypotheses. Table~\ref{tbl:gof-indices} summarises the results. The chi-square ($\chi^{2}$) value normalised by the degrees of freedom ($\chi^{2}/df$) was 2.34, below the recommended cut-off of 3. The GFI (0.92) and AGFI (0.93) exceeded the recommended threshold of 0.80, while the TLI (0.98) and CFI (1.12) were above the 0.90 threshold (note that AMOS occasionally reports CFI values slightly above 1.0 due to rounding). The RMSEA (0.07) was below the recommended maximum of 0.08. Overall, these results indicate that the structural model provides a good fit to the data \cite{hair_multivariate_2019}. 

\begin{table}[!htbp]
  \centering
  \fontsize{9.75pt}{10pt}\selectfont{}
  \renewcommand{\arraystretch}{1.8}
  \begin{tabular}{
    *{3}{>{\raggedright\arraybackslash}p{\dimexpr 0.333\linewidth-2\tabcolsep}}
  }
  \hline
  \textbf{Model Fit Indices} & \textbf{Recommended Value} & \textbf{Actual Value} \\
  \hline
  $\chi^2/df$ & $1 < NC < 3$ & $2.34$ ($p \leq 0.001$) \\
  RMSEA       & $< 0.08$     & $0.07$ \\
  GFI         & $> 0.80$     & $0.92$ \\
  AGFI        & $> 0.80$     & $0.93$ \\
  TLI         & $> 0.90$     & $0.98$ \\
  CFI         & $> 0.90$     & $1.12$ \\
  \hline
  \end{tabular}
  \caption{Research model's Goodness-of-Fit (GOF) indices}
  \label{tbl:gof-indices}
\end{table}

Table \ref{tbl:path-coefficients-and-hypothesis-testing-results} presents the hypothesis testing results from the structural model. The path coefficients ($\beta$) and significance levels (p-values) provide statistical support for eight hypotheses (H1, H2, H4, H5, H6, H7, H8, and H9), while there is insufficient support for H3. 

\begin{table}[!htbp]
  \centering
  \fontsize{9.75pt}{10pt}\selectfont
  \renewcommand{\arraystretch}{1.8}
  \begin{tabular}{
    >{\raggedright\arraybackslash}p{\dimexpr 0.18\linewidth-2\tabcolsep}  % Hypothesis
    >{\raggedright\arraybackslash}p{\dimexpr 0.06\linewidth-2\tabcolsep}  % Path (from)
    >{\centering\arraybackslash}p{\dimexpr 0.06\linewidth-2\tabcolsep}    % Arrow
    >{\raggedright\arraybackslash}p{\dimexpr 0.06\linewidth-2\tabcolsep}  % Path (to)
    >{\centering\arraybackslash}p{\dimexpr 0.23\linewidth-2\tabcolsep}    % Coefficient
    >{\centering\arraybackslash}p{\dimexpr 0.11\linewidth-2\tabcolsep}    % SE
    >{\centering\arraybackslash}p{\dimexpr 0.11\linewidth-2\tabcolsep}    % p
    >{\raggedright\arraybackslash}p{\dimexpr 0.19\linewidth-2\tabcolsep}  % Result
  }
  \hline
  \textbf{Hypothesis} & \textbf{Path} & & & \textbf{Coefficient ($\beta$)} & \textbf{SE} & \textbf{p} & \textbf{Result} \\
  \hline
  H1 & TC   & $\rightarrow$ & HM   & 0.43  & 0.065 & 0.003 & Supported \\
  H2 & TCs  & $\rightarrow$ & EE   & 0.18  & 0.067 & 0.000 & Supported \\
  H3 & PL   & $\rightarrow$ & TC   & 0.02  & 0.072 & 0.170 & Not Supported \\
  H4 & TC   & $\rightarrow$ & TTF  & 0.56  & 0.055 & 0.000 & Supported \\
  H5 & TCs  & $\rightarrow$ & TTF  & 0.66  & 0.049 & 0.005 & Supported \\
  H6 & PL   & $\rightarrow$ & TTF  & -0.33 & 0.061 & 0.003 & Supported \\
  H7 & HM   & $\rightarrow$ & TTF  & 0.26  & 0.054 & 0.000 & Supported \\
  H8 & EE   & $\rightarrow$ & TTF  & 0.41  & 0.060 & 0.000 & Supported \\
  H9 & TTF  & $\rightarrow$ & BI   & 0.33  & 0.057 & 0.000 & Supported \\
  \hline
  \end{tabular}
  \caption{Structural model path coefficients and hypothesis testing results}
  \label{tbl:path-coefficients-and-hypothesis-testing-results}
\end{table}

Figure \ref{fig:ttf-diagram-jise-results} presents the model from Figure \ref{fig:ttf-diagram-jise} with the corresponding path coefficients, standard errors, and significance levels for each hypothesised relationship in the extended TTF model. 

\begin{figure}[htbp]
\centering{}
\smallskip
\includegraphics[width=1.0\textwidth]{chapters/chapter04-publications/jise/ttf-diagram-jise-results}
\caption{Extended TTF model with path estimates and significance. Constructed by the authors.}
\label{fig:ttf-diagram-jise-results}
\end{figure}

The hypothesis testing results indicated strong support for most proposed relationships in the extended TTF model. 
\textbf{H1} -- Task characteristics positively influenced hedonic motivation in Thonny-py5mode-supported learning 
($\beta$ = 0.43, p = 0.003), indicating that engaging, well-structured tasks enhanced enjoyment and emotional engagement. 
\textbf{H2} -- Technology characteristics positively influenced effort expectancy 
($\beta$ = 0.18, p < 0.001), suggesting that user-friendly features reduced the perceived effort required to use Thonny-py5mode. 
\textbf{H3} -- Personalised learning did not significantly influence task characteristics 
($\beta$ = 0.02, p = 0.170), indicating that adaptive or individualised elements did not alter perceptions of the learning tasks. 
\textbf{H4} -- Task characteristics positively influenced task--technology fit 
($\beta$ = 0.56, p < 0.001), showing that clarity and relevance shaped perceptions of tool--task alignment. 
\textbf{H5} -- Technology characteristics positively influenced task--technology fit 
($\beta$ = 0.66, p = 0.005), underscoring the role of usability and functionality. 
\textbf{H6} -- Personalised learning negatively influenced task--technology fit 
($\beta$ = -0.33, p = 0.003), suggesting that greater personalisation sometimes reduced perceived alignment. 
\textbf{H7} -- Hedonic motivation positively influenced task--technology fit 
($\beta$ = 0.26, p < 0.001), affirming that emotional engagement improved perceptions of alignment. 
\textbf{H8} -- Effort expectancy positively influenced task--technology fit 
($\beta$ = 0.41, p < 0.001), indicating that ease of use strengthened perceived fit. 
\textbf{H9} -- Task--technology fit positively influenced behavioural intention to use Thonny-py5mode 
($\beta$ = 0.33, p < 0.001), showing that students were more likely to adopt the environment when it aligned with their learning needs.

\subsection{Discussion}
\label{subsec:discussion}

Our structural model (Figure \ref{fig:ttf-diagram-jise-results}) shows that task characteristics, technology characteristics, personalised learning, hedonic motivation, and effort expectancy all have direct, significant effects on the TTF of Thonny-py5mode. Moreover, this fit has a direct, significant effect on students' adoption of Thonny-py5mode for learning Python across TUA's four campuses in Australia and in its online cohort. However, personalised learning did not significantly affect technology characteristics. 

H1 tested whether task characteristics influence hedonic motivation, demonstrating that the model supports H1 ($\beta$ = 0.43, p = 0.003). This aligns with previous findings that well-structured, cognitively engaging learning tasks can significantly enhance learners' enjoyment. For instance, Škorić \textit{et al.} (2021) found that aligning programming task demands with technology capabilities improves student satisfaction and continued use intentions~\cite{skoric_exploring_2021}; similarly, Melzer (2019) noted that task design (an aspect of task characteristics) can strengthen motivation and engagement in information systems education~\cite{melzer_conceptual_2019}. In our context, Thonny-py5mode's support for step-by-step debugging, output visualisation, and structured logic-based graphical challenges appears to heighten students' emotional involvement, thereby raising hedonic motivation. These task effects also ripple into effort expectancy perceptions. When learners like using a tool because it structures tasks clearly and responds instantly, they tend to see the tool as less effortful. In this study, Thonny-py5mode matched the structure of students' tasks and supplied immediate feedback, increasing intrinsic motivation and helping create a more pleasant Python learning experience. 

H2 examined the link between technology characteristics and effort expectancy, confirming that Thonny-py5mode significantly influenced the latter for students learning Python programming ($\beta$ = 0.18, p = 0.007). Prior research finds that usability, clear feedback, and responsive system design reduce cognitive load and increase perceived ease of use. Wang and Kartika Sari (2024) reported that user-friendly interface features and visual feedback mechanisms enhance perceived ease of use in educational technology contexts~\cite{wang_examining_2024}; Škorić \textit{et al.} (2021) observed that web-based programming tools offering simplified navigation, real-time debugging, and platform accessibility support higher levels of perceived task--technology fit and reduce effort required~\cite{skoric_exploring_2021}. Thonny-py5mode's intuitive layout, error reporting, and graphical output lower the cognitive and physical effort students expend during learning. This pattern suggests that effort expectancy is shaped by how well the system aligns with novices' cognitive needs. When friction drops, learners can focus on grasping Python concepts rather than wrestling with the environment. 

H3 tested whether personalised learning shapes task characteristics. Contrary to some prior work, our model did not support H3 ($\beta$ = 0.02, p = 0.170). Melzer (2019) reported that personalisation, when grounded in cognitive fit theory, can improve learners' perceptions of task clarity in information systems education~\cite{melzer_conceptual_2019}. However, our findings suggest that personalisation alone does not automatically yield clearer or better-structured tasks. To influence task characteristics, system designers likely need to embed personalisation into the logic and scaffolding of the tasks themselves, not only into interface-level features. 

H4 concerned the impact of task characteristics on task--technology fit, demonstrating that the effect is significant and positive ($\beta$ = 0.56, p < 0.001). Prior research confirms that when tasks are clearly defined and supported by interactive affordances such as real-time feedback and visualisation, students perceive a stronger TTF~\cite{liu_influence_2023}. Yaakop \textit{et al.} (2020) found that collaborative and well-structured tasks in cloud-based learning systems improved perceived alignment between tasks and technological tools~\cite{yaakop_examining_2020}. Thonny-py5mode's syntax alerts, integrated visual feedback, and simplified debugging strengthen the perceived fit. 

H5 tested whether technology characteristics directly influence task--technology fit, showing a strong positive relationship ($\beta$ = 0.66, p < 0.005). Research from Thabet \textit{et al.} (2024) shows that system quality elements---particularly intuitive interfaces, real-time visual output, and error-handling assistance---improve perceptions of how well a technology matches task demands~\cite{thabet_exploring_2024}; similarly, Wang and Kartika Sari (2024) observed that platform responsiveness and adaptability enhance user engagement and strengthen TTF perceptions~\cite{wang_examining_2024}. In this study, the approachable Thonny-py5mode interface, convenient install process, graphical output, and shell panel support closely match the requirements of creative coding tasks. Through this alignment, students can focus on problem-solving without hindrance from the environment, thereby reinforcing perceptions of strong TTF. 

H6 indicates that personalised learning had a significant negative effect on task--technology fit ($\beta$ = −0.33, p = 0.003), suggesting that greater personalisation was associated with lower perceived alignment between Thonny-py5mode and the programming tasks. This aligns with H3's finding that personalisation did not significantly improve task characteristics, implying limited integration with the task structure. In contrast, Melzer (2019) found that tailoring environments to individual preferences can strengthen perceived fit~\cite{melzer_conceptual_2019}; Ishaq \textit{et al.} (2024) reported that adaptive difficulty, personalised feedback, and real-time code visualisation can enhance alignment in programming contexts~\cite{ishaq_level_2024}. One possible explanation is that aspects of Thonny-py5mode that could support personalisation, such as immediate visual feedback and the creative flexibility of py5 commands and graphical output, were not tailored to individual learners or sufficiently integrated into the assessment scaffolding. This may have made them seem loosely connected to the programming tasks and fostered a perception of mismatch rather than synergy. 

H7 addresses hedonic motivation's role in task--technology fit, indicating a significant and positive relationship ($\beta$ = 0.26, p < 0.001). Škorić \textit{et al.} (2021) found that when students perceive interactive programming environments as enjoyable, their assessment of TTF improves~\cite{skoric_exploring_2021}; Ishaq \textit{et al.} (2024) similarly reported that adaptive difficulty, personalised feedback, and real-time code visualisation can heighten both enjoyment and perceived alignment in programming contexts~\cite{ishaq_level_2024}. Thonny-py5mode's graphics-oriented output provides a form of immediate visual feedback (alongside shell output), and its simplified interface likely make programming less intimidating for Python novices. 

H8 tested the effect of effort expectancy on task--technology fit, showing a significant and positive relationship ($\beta$ = 0.41, p < 0.001). Prior research supports this link between perceived ease of use and stronger perceptions of task--technology alignment. Lim and Lee (2021) found that intuitive interfaces and reduced complexity enhance users' sense of fit with learning tasks~\cite{lim_extended_2021}; Thabet \textit{et al.} (2024) reported that clear navigation structures, immediate feedback, and low cognitive load significantly improve the perceived match between system capabilities and intended tasks~\cite{thabet_exploring_2024}. Thonny-py5mode's clean layout, integrated debugging, and graphical output likely reduced effort, reinforcing perceptions of fit with the graphical programming tasks. 

H9 links task--technology fit to behavioural intention; the effect is significant and positive ($\beta$ = 0.33, p < 0.001). DeLone and McLean's (2003) IS success model emphasises the role of system--task alignment in driving continued use~\cite{delone_delone_2003}; Škorić \textit{et al.} (2021) confirmed that when students perceive web-based programming tools as fitting well with learning activities, they show stronger continuance intentions~\cite{skoric_exploring_2021}.  

Taken together, these results highlight an integrated picture of adoption. Task--technology fit is a central driver of behavioural intention, shaped by clear, well-scaffolded tasks (H1, H4), strong technology characteristics (H2, H5), and user-centric elements that make the experience both enjoyable (H7) and easy (H8). In Thonny-py5mode, features such as syntax alerts, intuitive debugging, and integrated graphical output help align the environment closely with the demands of graphical Python tasks. Personalised learning, however, shows a more complex pattern: it does not significantly alter task characteristics (H3) and, in this study, was associated with lower TTF (H6). This suggests that when adaptive elements, such as immediate visual feedback or the creative possibilities offered by py5 commands, are not tightly embedded in task scaffolding, they may reduce rather than enhance perceived alignment. 
 
For practice, three implications stand out: (1) design tasks that are structured, cognitively engaging, and tightly mapped to learning outcomes, while ensuring the environment provides immediate, meaningful feedback; (2) adopt environments with novice-friendly interfaces, visual feedback, and streamlined debugging features that reliably lower effort, raise enjoyment, and strengthen TTF; and (3) treat personalisation with care: ensure that task scaffolding and learning pathways closely integrate with adaptive pacing and feedback, rather than layering these superficially at the interface level. 

In the specific case of Thonny-py5mode, the convergence of usability, visual immediacy, and supportive debugging appears to deliver a strong perceived fit to core Python programming tasks. Students who experienced this fit reported greater enjoyment, lower perceived effort, and stronger intentions to continue using the environment. The overall pattern suggests that Processing-like, graphically-oriented Python environments such as Thonny-py5mode can meaningfully enhance early programming experiences when they balance clear task design, robust interactive affordances, and thoughtfully embedded personalisation. 

\subsection{Research Implications}

The study offers both theoretical and practical contributions for the use of Thonny-py5mode. Theoretically, it extends TTF theory by incorporating three personal characteristics: personalised learning, hedonic motivation, and effort expectancy. While traditional TTF frameworks focus on aligning task needs and technology characteristics~\cite{dwivedi_task-technology_2012}, our model posits that learners' experiences and motivations also mediate this relationship. Testing hypotheses 1 through 9, we confirm that perceived fit depends on both system functionality and users' enjoyment, learning needs, and intention to use the system, echoing prior work demonstrating the joint influence of system quality and user factors on TTF~\cite{skoric_task-technology_2021, thabet_exploring_2024}, thereby expanding TTF to include emotional and motivational dimensions alongside functional alignment. The insignificant effect between personalised learning and technology characteristics further suggests that adaptive learning systems must consider not only the quality of system features but also the user's context and capability, consistent with insights from Melzer (2019) on the role of cognitive fit in personalised learning design~\cite{melzer_conceptual_2019}. Design implications resonate with findings that effective curriculum transformation requires embedding authentic uses of technology in programme design \cite{rambe_role_2016}. That sustained self-efficacy support through hands-on, scaffolded training enables educators to adopt more constructive, knowledge-building uses of ICT in classrooms \cite{mlambo_effects_2020}.

Practically, the findings suggest selecting or designing coding environments like Thonny-py5mode for their ability to support structured learning tasks while remaining stimulating and user-friendly. In Thonny-py5mode, features such as graphical output via py5 sketches, seamless integration between the sketch window and Thonny panels, and a simplified, non-command-line interface that avoids package management reduce cognitive load and boost satisfaction~\cite{lim_extended_2021, thabet_exploring_2024}. These elements align closely with our results on task--technology fit, as the environment's layout and immediacy in generating graphics help beginners stay oriented and engaged. The strong link between hedonic motivation and TTF is illustrated in Thonny-py5mode's ability to make coding sessions enjoyable through real-time, graphical results, encouraging sustained use~\cite{skoric_exploring_2021, yaakop_examining_2020}. For educators, these findings can inform the adaptation of existing IDEs or the design of new ones tailored to specific Python multimedia libraries, without sacrificing ease of use. By pairing beginner-friendly tools with scaffolded coding tasks and challenges, educators can help novices build confidence and skills. Finally, the significant influence of TTF on behavioural intention shows that when students feel Thonny-py5mode fits their needs, they are more inclined to continue using it; this underscores the importance of aligning the tool's features with both pedagogical goals and learner characteristics~\cite{delone_delone_2003}.

\subsection{Conclusion}

This study employed an extended Task--Technology Fit framework to investigate factors influencing university students' intention to use Thonny-py5mode for learning Python. The findings demonstrate that task characteristics, technology characteristics, hedonic motivation, effort expectancy, and personalised learning all contributed to perceived fit, with TTF subsequently driving behavioural intention. While personalisation did not significantly affect task characteristics and was, in this case, associated with lower TTF, other factors proved more influential. The integration of structured tasks, a user-friendly design, and an engaging graphical approach fostered strong alignment between Thonny-py5mode and learners' needs.

For practice, our findings underscore the value of beginner-friendly IDEs that integrate graphical output and beginner-friendly design, complemented with scaffolded tasks to reduce cognitive load and sustain engagement. We demonstrated this through extending TTF theory, integrating the role of motivational constructs alongside functional alignment in predicting technology adoption for programming education. In Thonny-py5mode, a balance of usability and visually fostered enjoyment, lowered effort, and encouraged continued use. Future research should explore how to embed personalisation in task scaffolding more tightly, and how contextual factors such as prior programming experience and educator support influence adoption outcomes. 
