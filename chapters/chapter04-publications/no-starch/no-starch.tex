\subsection{Background \& Context}

\textit{Learn Python Visually} (Figure~\ref{fig:book-cover-photo}) represented a significant component of this PhD research, concluding the initial phase of investigation into- and development of novel Python-based creative coding techniques. Aimed at beginner coders and readers transitioning from other programming backgrounds, the book employs Processing's Python Mode (Processing.py) as its pedagogical foundation, combining Python's accessibility with Processing's multimedia capabilities.

\begin{figure}[htbp]
\centering
\includegraphics[width=1\textwidth]{chapters/chapter04-publications/no-starch/book-cover-photo}
\caption{\textit{Learn Python Visually: Creative Coding with Processing.py}. Photo by the author.}
\label{fig:book-cover-photo}
\end{figure}

The book's strength and novelty lie in its visual and practical take on Python instruction. Each chapter builds progressively upon the last, moving from fundamental concepts toward more sophisticated creative coding techniques. This approach directly aligns with the overarching aim of the thesis: enhancing Python programming education through creative computing environments, and specifically the \hyperref[ro:educational-materials]{Educational Materials} and \hyperref[ro:broader-dissemination]{Broader Dissemination} objectives. Two presentations in particular, \nameref{sec:processing.py-creative-coding-with-python} and \nameref{sec:processing-python-mode-for-creative-coding-and-teaching}, provide further insight into the activities and research that informed the book's development. 

No Starch Press is a respected independent publisher specialising in technical and educational titles. Established in 1994, it has built a strong reputation for publishing accessible yet rigorous works on programming, cybersecurity, and STEM education, including bestsellers such as \textit{Python Crash Course}, \textit{Python for Kids}, \textit{How Linux Works}, and \textit{Hacking: The Art of Exploitation}. Its titles---widely adopted by learners, educators, and professionals---have received industry recognition, including Independent Publisher Book Awards (from \textit{Independent Publisher} magazine) and inclusion in the prestigious \textit{Communication Arts Design Annual}. This status underscores the book's credibility as both a practical learning resource and a contribution to Python-based programming education~\cite{no_starch_press_about_2025}.

Paddy Gaunt, who served as the book's technical reviewer, studied engineering at Cambridge University (UK). Since 2012, he has maintained pi3d: a Python library designed to deliver high-speed 3D graphics on Raspberry Pi microcomputers~\cite{bunn_learn_2021}.

\textit{Learn Python Visually} is available worldwide in both print and digital formats. It has received favourable reviews on Amazon.com\footnote{~\url{https://www.amazon.com/Learn-Python-Visually-Tristan-Bunn/dp/1718500963}} where it features editorial endorsements from Saber Khan (Education Community Director of the Processing Foundation), Dr.~Ralf Biedert (Principal Engineer at Tobii AB), and Alfred Abusomwan (Techs Blog). As of June 2025, the book had achieved 2,622 direct sales (Appendix \ref{appendix:no-starch-royalty-statement}). It was also included in two Humble Bundle collections: \textit{Machine Learning by No Starch Press} (August 2021) and \textit{Python by No Starch Press} (May 2023), that sold 17,669 and 10,881 bundles, respectively~\cite{humble_bundle_inc_machine_2021, humble_bundle_inc_humble_2023}. 

In addition to direct and bundled sales, the book is distributed internationally through No Starch Press partners, including digital platforms such as O'Reilly Online Learning\footnote{~\url{https://www.oreilly.com/library/view/learn-python-visually/9781098128937}} and through academic library collections, further extending its educational reach. Penguin Random House distributes No Starch Press printed titles in the USA and worldwide~\cite{no_starch_press_sales_2025}.

\textit{Learn Python Visually} integrates visual learning throughout its chapters, requiring learners to program output that expresses computational methods through graphical outputs, animations, data visualisations, 2D simulations, and interactive interfaces. In this way, it places emphasis on immediate, observable results to elucidate abstract concepts and support intuitive understanding. The book is printed in full colour, incorporating visual explanations in addition to the results of different code samples. Figures~\ref{fig:book-p190-p205} and~\ref{fig:book-p210-p225} present a sample selection of 16 pages each.

\begin{figure}[htbp]
\centering
\includegraphics[width=\textwidth]{chapters/chapter04-publications/no-starch/book-p190-p205}
\caption{Excerpt from the author's book \textit{Learn Python Visually} (pp.~190--205), published by No Starch Press. This section introduces trigonometric functions and waveforms, and demonstrates the generation of Lissajous figures using Python. Source:~\cite{bunn_learn_2021}.}
\label{fig:book-p190-p205}
\end{figure}

\begin{figure}[htbp]
\centering
\includegraphics[width=\textwidth]{chapters/chapter04-publications/no-starch/book-p210-p225}
\caption{Excerpt from the author's book \textit{Learn Python Visually} (pp.~210--225), published by No Starch Press. This section introduces object-oriented principles through the generation of an amoeba simulation using Python. Source:~\cite{bunn_learn_2021}.}
\label{fig:book-p210-p225}
\end{figure}

The early chapters cover introductory concepts, such as installation procedures, basic syntax, arithmetic operations, and the principles of drawing with code. Once readers are familiar with the fundamentals, later chapters build on these foundations, shifting focus to concepts that require more programmatic reasoning---control flow, iteration, randomness, motion, transformations, data visualisation, and constructing a user interface. The following outline provides a brief overview of each chapter's contents:

\begin{quote}

\noindent\textbf{Chapter 1: Hello, World!} \\
Covers installation and setup for Processing Python Mode (Processing.py) and introduces the basics of drawing with code. Also: how computers manage colour, how to store and reuse values using variables, and how to perform basic arithmetic operations in Python.

\vspace{0.5em}
\noindent\textbf{Chapter 2: Drawing More Complicated Shapes} \\
Building on drawing essentials, readers move on to organic (non-geometric) shapes---defining these with points (vertices) and curves, which can describe almost any 2D form in code.

\vspace{0.5em}
\noindent\textbf{Chapter 3: Introduction to Strings and Working with Text} \\
Introduces Python's string features for manipulating text. To visualise these concepts, readers explore Processing functions for drawing text to the display window in different styles, colours, and fonts.

\vspace{0.5em}
\noindent\textbf{Chapter 4: Conditional Statements} \\
Moving into control flow, demonstrating how to write programs that make decisions and execute different actions in response to different situations. Again, readers explore these concepts through graphical output.

\vspace{0.5em}
\noindent\textbf{Chapter 5: Iteration and Randomness} \\
Covers repeating operations in Python, either a specified number of times or until a condition is met. Readers combine these techniques with randomness to generate tiled patterns.

\vspace{0.5em}
\noindent\textbf{Chapter 6: Motion and Transformation} \\
Introduces the \texttt{draw()} function for animated output. Also covers saving frames as images, transforming groups of visual elements, and employing real-time values for screensavers and clock displays.

\vspace{0.5em}
\noindent\textbf{Chapter 7: Working with Lists and Reading Data} \\
Unlocks techniques for managing and manipulating collections of values, leading to practical data visualisation examples that read in external CSV data to generate charts dynamically.

\vspace{0.5em}
\noindent\textbf{Chapter 8: Dictionaries and JSON} \\
Explores richer data structures and visualisation with JSON data, developing readers' skills to work with more complex and novel data visualisations.

\vspace{0.5em}
\noindent\textbf{Chapter 9: Functions and Periodic Motion} \\
Shows how dividing programs into reusable functions improves modularity and readability, then combines these techniques with trigonometric operations to generate elliptical and wave-type motions.

\vspace{0.5em}
\noindent\textbf{Chapter 10: Object-Oriented Programming and PVector} \\
Demonstrates object-oriented principles to structure programs that model real-world phenomena, guiding readers through programming a visually engaging amoeba simulation that leverages vector-based motion.

\vspace{0.5em}
\noindent\textbf{Chapter 11: Mouse and Keyboard Interaction} \\
Concludes the book by adding interactivity to Processing sketches, focusing on mouse and keyboard input to build a paint app. This requires event functions and techniques for controlling Processing's draw-loop behaviour.

\end{quote}

An Afterword discusses where readers might head next, pointing them to the Processing community forum and useful materials, including tutorials on working with Images \& Pixels, P3D for 3D work, and the broader Processing ecosystem of libraries (written for Java Mode but typically portable to Python Mode). It encourages translating Java examples, highlights Daniel Shiffman's \textit{The Nature of Code} (with Python ports), and discusses Python's applications in games, the web, data, and AI. Additionally, there is some discussion on adjacent creative-coding environments, including p5.js, JRubyArt, openFrameworks, OPENRNDR, and hardware programming with Arduino.

Notably, readers' prior experience in related areas, including other programming languages, will likely influence their comprehension and the pace at which they progress through the book. However, the text encourages detours wherever inspiration strikes and includes challenges and practical tasks at key moments to reinforce conceptual understanding and promote active learning. For complete beginners, the content aims to reduce the anxiety often associated with textual coding by incorporating visual tasks wherever possible.

\subsection{Conclusion}

Within the framework of this thesis, \textit{Learn Python Visually} consolidates extensive research into Python creative computing, specifically the Processing.py environment, translating experimental work into pedagogical strategies and learning materials. Publication by No Starch Press (est.~1994), with technical review by Paddy Gaunt (maintainer of pi3d), attests to rigour. Uptake indicators include 2,492 direct sales as of Dec~2024, inclusion in two Humble Bundle collections that together sold more than 25,000 digital copies, and worldwide distribution. Collectively, these figures evidence effective pedagogy and and broad accessibility.
